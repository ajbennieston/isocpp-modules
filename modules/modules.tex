\documentclass[reqno]{article}
\usepackage{fontspec}
\usepackage[english]{babel}
\usepackage[reqno]{amsmath}
\usepackage{amssymb}
\usepackage[utf8]{inputenc}
\usepackage[
    paper=a4paper,
    textwidth=15cm,
    textheight=24cm,
    ]{geometry}
\usepackage{fancybox}
\usepackage{url}
\usepackage[
    breaklinks,
    citecolor=blue,
    colorlinks,
    linkcolor=blue,
    urlcolor=blue,
    ]{hyperref}
\usepackage{minted}
\usepackage{color}
\usepackage{graphicx}
\usepackage{epstopdf}
\usepackage{fancyhdr}
\usepackage[
    comma,
    numbers,
    sort&compress,
    square,
    ]{natbib}
\usepackage{amsthm}
\usepackage{multicol}

\title{What do we talk about when we talk about Modules}
\author{Gašper Ažman \small{(gasper.azman@gmail.com)}
\and Andrew Bennieston \small{(a.j.bennieston@gmail.com)}}
\date{\today}

\begin{document}
\maketitle
\section{Aim of this paper}

The Modules TS \citep{N4647} has generated much discussion. This paper aims to
capture the questions and decisions that are necessary to arrive at a coherent
proposal.

This paper takes no stance on any of the issues highlighted. Where possible,
the differing opinions are cited and summarised in the discussion of each
section. It is likely that there are issues and discussions of which the
authors are not aware. We do not claim to represent every facet of the design
space, and expect further iteration for refinement.

The goal of the authors of this paper is to disentangle the various issues
people have with modules, so that constructive argument can happen along the
issues and requirements. The issues are presented in as orthogonal a manner as
possible, so that loops in arguing are avoidable.

\subsection{A note on terminology}

This paper uses the word \emph{name} in several contexts. It can be taken to
mean \emph{id-expression} as defined in [expr.prim.id]~\citep{N4659} or can
be extended to include the names of macros. We attempt to call out the specific
meaning in the cases where it is important. We also use the term \emph{entity}
as defined in \texttt{[basic]}.~\citep{N4659}

\section{What Are Modules For?}

This section presents all the various answers to the titular question that the
authors are aware of. It is probable that not all the answers are mutually
compatible, realistic, or even on topic. However, unless stated and
subsequently addressed, they will keep surfacing.


\subsection{Reducing the need for detail namespaces}

Name lookup is a flexible and powerful feature of C++, but the rules are
complex and it is no secret that the set of names (including names of macros)
that are visible in user code is often larger than would be desirable.
Customisation points further complicate matters.

To do this, private (and, increasingly often, public) features of libraries are
implemented in a library-specific detail namespace, with externally visible
names pulled into the public-facing main namespace with \texttt{using}
directives.

This keeps implementations sane, but has the unfortunate side-effect of making
compiler diagnostic messages longer than would be ideal.

One view is that it would be nice for modules to obviate the need to use detail
namespaces to hide names by allowing name lookup outside the module to only
find entities that are explicitly exported.


\subsection{Strict ordering guarantees for code inclusion}

Another issue that keeps library authors awake at night is the introduction of
new names into their namespaces due to code like

\begin{minted}{cpp}
#include "otherlibrary.hpp"
#include "mylibrary.hpp"
\end{minted}

where \texttt{otherlibrary} defines some names (or macros) that clash with
\texttt{mylibrary}.

An example of such a problem is the implementation of the standard library; all
names in it need to use the \texttt{\_\_name} and \texttt{\_Name} conventions,
despite already being in the \texttt{std} namespace -- macros are not allowed
to redefine names like that, since they are reserved. Regular libraries deal
with this problem by assuming it won't happen.

If modules were specified to guarantee a strict partial ordering, such that
\emph{only the code that happened\footnote{We use ``happened'' descriptively,
because it means different things depending on how the other features of
modules end up being specified. In the first instance, one could define it as
simple textual inclusion of all (and \emph{only}) transitive dependencies of a
module. Since those could also be modules in their own right, the actual
specification of this feature is difficult, and so we cautiously use the word
``happened''.} above this module is the code it specifically imports} --
possibly applied \emph{non}-transitively for name visibility -- then this issue
disappears, and the standard library (along with every other library) can stop
worrying about macros rewriting its code.

\subsection{Transitivity of name visibility}

Should \texttt{import moduleA;} also import the names from modules imported by
\texttt{moduleA} transitively, or not?

\subsubsection{Pro transitive}

The ``better-precompiled-headers'' approach dictates transitive name
visibility. Namespaces remain the (only) method by which name visibility is
managed, and modules are completely orthogonal.

\subsubsection{Pro intransitive}

The existing practice from almost all other languages dictates non-transitivity
of name visibility. This would mean explicit module imports everywhere, with
allowance for re-exporting to enable \emph{curated} library APIs.

\subsection{Replace Precompiled Headers: the \emph{Parse It Once} Argument}

C++ code takes a long time to parse, and longer to compile. There are many
reasons for this, but having to read millions of lines of files to just get the
declarations and definitions of all the visible, invisible, and irrelevant
names, for \emph{every compilation unit}, is one of them.

If there was a way to parse a header once, expose a data structure that would be
able to load only the needed code at the time it is needed, and share that
effort, the hope is that compilation times would drop.


\subsection{Replace Precompiled Headers: the \emph{Highlander} Argument}

Precompiled headers are also very inflexible, since \emph{there can only be
one}, and therefore encourage having a pinch-point in the compilation graph
that necessitates the recompilation of everything on every minor change of any
header.

The hope is that modules would act as mix-and-match precompiled headers,
keeping compilation quick while maintaining strict ``as needed'' import policies.


\subsection{Better Namespaces: Module Namespaces}

Some have the wish that modules would be better namespaces -- modules would
introduce namespaces, completing the C++-is-a-better-python dream\footnote{The
authors make no claim as to whether this is a dream worth having.}.

This goal is complicated by the necessity for code (which may be modularised)
to define entities in other namespaces. Standard library customisation points
serve as a prime example; it is desirable for a module to be able to define a
specialisation of e.g. \texttt{std::hash}, which suggests that modules and
namespaces may need to be orthogonal.

Although the concepts may be orthogonal, many expect that there will be a
strong correlation between modules and (outer) namespaces. The following ideas
have been proposed to help with the common cases.


\subsubsection{Introduce a namespace by default and offer an opt-out.}

A module can introduce a namespace by default, and then offer ways of
``escaping'', for instance:

\begin{minted}{cpp}
// All names implicitly in the foo::bar namespace and foo::bar module.
module foo::bar;

struct mystruct; // foo::bar::mystruct;

::std::hash<mystruct> {
  std::size_t operator(mystruct const&)
      const;
};
\end{minted}

This style of opt-out was specified in a paper by Tristan Bridle.~\citep{P0665R0}


\subsubsection{Module-namespaces: Introduce a scope}

A special syntax would enable introducing the module and its namespace together:

\begin{minted}{cpp}
module namespace foo::bar {
  // All names in the foo::bar namespace and foo::bar module.
  struct mystruct;
} // end module foo::bar

module namespace foo::baz {
  // Look ma, two modules per file!
  struct mystruct;
}

// Entities defined here are only visible within the file and cannot be exported.
namespace std {
struct std::hash<foo::bar::mystruct>
{
   std::size_t operator(
         foo::bar::mystruct const&)
       const;
};
} // namespace std
\end{minted}


\subsection{Better Namespaces: visibility qualifiers on names}

Since modules seem to have the ambition of offering tight control of entities,
they pose a question about what exactly they expose. The space of options
follows. These are mostly mutually incompatible -- macros are dealt with in a
separate section.


\subsubsection{Expose names}

In this way, once you've exported a name, it's visible, along with all its
overloads (for functions) or specializations (for types).


\subsubsection{Same as class specifiers}

Have access specifiers \texttt{public}, \texttt{private} and \texttt{protected}
work for modules the same way as they work for classes: expose names for the
purposes of overload set resolution, but break if the resolved-to entity is
not accessible (e.g. the name is private and the call is outside the module).


\subsubsection{Expose entities}

Only expose entities, explicitly. From outside the module, the private entities
are hidden from all name lookup (i.e. non-exported function names do not appear
in the overload set).


\subsection{Are macros exported?}

This is quite possibly the biggest current disagreement. This section summarises
the arguments and counter-arguments.


\subsubsection{Pro: Macros do appear in public interfaces of libraries}

From \texttt{MAX\_SIZE} to \texttt{assert} to \texttt{BOOST\_HANA\_STRING},
macros are a part of the public interface of many libraries, and therefore
modules should offer a way to expose them, goes the argument.

A simple workaround is to supply the macro interface of the library in a
separate header meant for the \texttt{include} mechanism. This header would
also import the non-macro part of the interface it depends on. This makes the
way modules would work with the preprocessor far more obvious.

However, this makes the general answer to \emph{how do I use libraries} a lot
\emph{less} obvious; the answer may depend on whether you (or your clients) use
the macros, or prefer a clean interface. Further, you may wish to consume the
macros internally, but hide them from your clients. 

The advantage is that libraries that are \texttt{import}ed will \emph{never}
leak macros, which is a strong guarantee that is nice to have.

\subsubsection{Contra (part 1): It breaks the compilation model}

If macros are exported, then how can the import directive run after the
preprocessor?


\subsubsection{Contra (part 2): I want to guarantee my import doesn't rewrite my source code}

This is a valid concern, but can probably be solved independently of macro
exports by looking at strict ordering guarantees for inclusion. It is assumed
that you can trust the libraries you import explicitly.


\section{Acknowledgements}

We would like to thank Vittorio Romeo for a number of helpful suggestions,
Jackie Kay for highlighting readability concerns and style suggestions, and
Louis Dionne for proofreading the paper.

\bibliographystyle{amsplain}

\begin{thebibliography}{9}

\bibitem{N4647}
    Gabriel Dos Reis.
        \textit{Working Draft, Extensions to C++ for Modules (N4647)}
        \url{http://www.open-std.org/jtc1/sc22/wg21/docs/papers/2017/n4647.pdf}

\bibitem{N4659}
    ISO/IEC SC22 WG21
        \textit{Working Draft, Standard for Programming Language C++}
        \url{http://www.open-std.org/jtc1/sc22/wg21/docs/papers/2017/n4659.pdf}

\bibitem{P0665R0}
    Tristan Brindle.
        \textit{Allowing Class Template Specialisations in Unrelated Namespaces}
        \url{https://github.com/tcbrindle/specialization_proposal/blob/master/P0665r0.pdf}

\end{thebibliography}

\end{document}
