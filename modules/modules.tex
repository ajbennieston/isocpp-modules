\documentclass[reqno]{article}
\usepackage[english]{babel}
\usepackage[reqno]{amsmath}
\usepackage{amssymb}
\usepackage[utf8]{inputenc}
\usepackage[
    paper=a4paper,
    textwidth=15cm,
    textheight=24cm,
    ]{geometry}
\usepackage{fancybox}
\usepackage{url}
\usepackage{minted}
\usepackage{color}
\usepackage{graphicx}
\usepackage{epstopdf}
\usepackage{fancyhdr}

\usepackage{amsthm}
\usepackage{multicol}

\title{What do we talk about when we talk about Modules}
\author{Gašper Ažman \small{(gasper.azman@gmail.com)}
\and Andrew Bennieston \small{(a.j.bennieston@gmail.com)}}
\date{\today}

\begin{document}
\maketitle
%\begin{multicols}{2}
\section{Aim of this paper}

This paper aims to be the definitive guide to arguing about Modules. To do
that, it presents all the questions and decisions that need to be taken in
order to arrive at a coherent proposal.

This paper takes no stance on any of the issues highlighted. Where possible, the
differing opinions are cited and summarized in the summary section for the
question.

The hope of the authors of this paper is to untangle the various issues people
have with modules, so that constructive argument can happen along the issues,
and not around FUD. The issues are presented in as orthogonal a manner as
possible, so that loops in arguing are avoidable.


\section{What Are Modules For?}

This section presents all the various answers to the titular question that the
authors are aware of. It is highly probable that not all the answers are
mutually compatible, realistic, or even on topic. However, unless stated and
subsequently addressed, they will keep surfacing.


\subsection{Reducing the need for detail namespaces}

Name lookup is a highly convoluted, flexible, and powerful feature of C++. It
is no secret that keeping the set of visible names sane in user code while
somehow allowing customization points keeps library authors awake at night. 

To do this, private (and more and more often) public features of libraries are
implemented in a library-specific detail namespace, with externally visible
names pulled into the public-facing main namespace with \texttt{using}
directives.

This keeps implementations sane, but compiler error messages then include the
detail namespace, making names longer than needed.

It would be nice for modules to obviate the need to use detail namespaces to
hide names by allowing name lookup outside the module to only find entities that
are explicitly exported.


\subsection{Strict ordering guarantees for code inclusion}

Another issue that keeps library authors awake at night is the introduction of
new names into their namespaces due to code like

\begin{minted}{cpp}
#include "otherlibrary.hpp"
#include "mylibrary.hpp"
\end{minted}

where \texttt{otherlibrary} defines some names that clash with
\texttt{mylibrary}.

An example of such a problem is the implementation of the standard library! This
is one of the reasons all names in it need to use the \texttt{\_\_name} and
\texttt{\_Name} conventions, despite already being in the \texttt{std} namespace
- macros are not allowed to redefine names like that, since they are reserved.

Regular libraries deal with this problem by assuming it won't happen and going
“la la la this is real-world code”.

If modules specified a strict ordering guarantee of \emph{the only code that
“happened” above this module is the code it specifically imports - and that goes
NON-transitively for name visibility}, this is no longer an issue, and the
standard library, along with every other library, can stop worrying about macros
rewriting its code.


\subsection{Replace Precompiled Headers: the \emph{Parse It Once} Argument}

C++ Code takes a long time to parse, and longer to compile. There are many
reasons for this, but having to read millions of lines of files to just get the
declarations and definitions of all the visible, invisible, and irrelevant
names, for \emph{every compilation unit}, is one of them.

If there was a way to parse a header once, expose a data structure that would be
able to load only the needed code at the time it is needed, and share that
effort, the hope is compilation times would drop.


\subsection{Replace Precompiled Headers: the \emph{Highlander} Argument}

Precompiled headers are also very inflexible, since \emph{there can only be
one}, and therefore encourage having a pinch-point in the compilation graph that
necessitates the recompilation of everything on every minute change of any
header.

The hope is modules would act as mix-and-match precompiled headers, keeping
compilation quick while maintaining strict “as needed” import policies.


\subsection{Better Namespaces: Module Namespaces}

Some have the wish that modules would be better namespaces - modules would
introduce namespaces, completing the C++-is-a-better-python dream\footnote{The
authors make no claim as to whether this is a dream worth having}.

Unfortunately for this one, it is impossible - modules need to be able to
provide specializations of \texttt{std::hash}, at least, which necessitates the
orthogonality of modules and namespaces.

However, some expect that modules and namespaces, while orthogonal, will be
substantially \emph{correlated}. They propose the following ideas to help the
common cases.


\subsubsection{Introduce a namespace by default and offer an opt-out.}

A module can introduce a namespace by default, and then offer ways of
“escaping”, for instance:

\begin{minted}{cpp}
// all names implicitly in the
// foo::bar namespace
module foo::bar;

struct mystruct; // foo::bar::mystruct;
::std::hash<mystruct> {
  std::size_t operator(mystruct const&)
      const;
};
\end{minted}

The opt-out has been specified in a paper by Tristan Brindle: \emph{Allowing
Class Template Specializations in Unrelated Namespaces}\footnote{
\url{https://github.com/tcbrindle/specialization_proposal/blob/master/P0665r0.pdf}}
.

\subsubsection{Module-namespaces: Introduce a scope}

A special syntax would enable introducing the module and its namespace together:

\begin{minted}{cpp}
module namespace foo::bar {
  // all names in the foo::bar namespace
  struct mystruct;
} // end module foo::bar

module namespace foo::baz {
  // look ma, two modules per file!
  struct mystruct;
}
// Entities defined here are
// only visible within the
// file and cannot be exported.
namespace std {
struct std::hash<foo::bar::mystruct>
{
   std::size_t operator(
         foo::bar::mystruct const&)
       const;
};
} // std
\end{minted}


\subsection{Better Namespaces: visibility qualifiers on names}

Since modules seem to have the ambition of offering tight control of entities,
they pose a question about what exactly they expose. The space of options
follows. These are mostly mutually incompatible, except for the whole macros
story, which has a separate section completely.

\subsubsection{Expose names}
In this way, once you've exported a name, it's visible, along with all its
overloads (for functions) or specializations (for types).

\subsubsection{Same as class specifiers}
Have access specifiers \texttt{public}, \texttt{private} and \texttt{protected}
work for modules the same way as they work for classes: expose names for the
purposes of overload set resolution, but break if the resolved-to entity is
not accessible (eg. private and the call is outside the module).

\subsubsection{Expose entities}
Only expose entities, explicitly. From outside the module, the private entities
do not even appear in the overload set.


\subsection{Are macros exported?}

This is quite possibly the biggest current disagreement. This section summarizes
the arguments and counter-arguments.


\subsubsection{It breaks the compilation model}

If macros are exported, then how can the import directive run after the
preprocessor?

\subsubsection{I want to guarantee my import doesn't rewrite my source code}

It is a valid concern, but can be solved independently of macro exports by
looking at strict ordering guarantees for inclusion. Presumably you can trust
the libraries you include explicitly.

\subsubsection{Macros do appear in public interfaces of libraries}

From \texttt{MAX\_SIZE} to \texttt{assert} to \texttt{BOOST\_HANA\_STRING},
macros are a part of the public interface of many libraries, and therefore
modules should offer a way to expose them, goes the argument.

A possible solution (if your answer is still \emph{no}) is to offer the macros
in a separate header meant for the \texttt{include} mechanism that imports the
non-macro part of the interface it depends on. This makes the way modules would
work with the preprocessor far more obvious, but on the flip side makes the
answer to \emph{how do I use this library} a whole lot \emph{less} obvious.
Again, on the flip side, it means libraries that are \texttt{import}ed will
\emph{never} leak macros.

\section{Acknowledgements}
We would like to thank Vittorio Romeo for a few suggestions, and Jackie Kay and
Louis Dionne for proofreading the paper.

%\end{multicols}
\bibliographystyle{amsplain}
% \bibliography{mybibliography}
\end{document}
